\section{Introduction}
\label{sec:introduction}
Arduino designet finnes her:  https://www.tinkercad.com/things/kvRdZLvI430 \\
Hele Oblig 4 finnes her: https://github.com/h180339/DAT110-Oblig4-DelB\\

Systemet består av en simulert Arduino, en Java simulert Arduino, og en Java Spark tjenerapplikasjon. Arduinoen er programmert og satt opp til å være en enkel adgangskontrollenhet. Den kan låses opp om det trykkes inn riktig kode på knappene. Og vil så etter en satt tid gå tilbake til låst automatisk. 

Tjenerapplikasjonen tilbyr et REST API for IoT devicer som kan lagre og hente informasjon fra clouden. Dette oppnår vi med å bruke HTTP protokollens GET og POST operasjoner for å sende Json meldinger. Vi bruker Spark rammeverket for å lage enkle "servlets" som behandler innkommende HTTP requests. 

Arduinoen er koblet opp med en PIR detektor, 2 impuls knapper og 3 status dioder. Denne er simulert i både TinkerCad og Java, dette er fordi TinkerCad ikke tilbyr nettverksfunksjonalitet for den. Derfor er implementasjonen i Java den som blir brukt mot Cloud tjenesten. Koden her sender enkle HTTP requests til tjeneren for å logge, samt henter ned nye tilgangskoder som er satt.

\pagebreak