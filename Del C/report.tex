\documentclass[11pt]{article}

\usepackage{a4wide}
\usepackage{mathptm}
\usepackage{xspace}
\usepackage{amsmath}
\usepackage{graphicx}
\usepackage{algorithm}
\usepackage{algpseudocode}
\usepackage{tikz}
\usepackage{tkz-graph}
\usetikzlibrary{shapes.misc, positioning}
\usepackage{listings}
\usepackage{color}

\definecolor{dkgreen}{rgb}{0,0.6,0}
\definecolor{gray}{rgb}{0.5,0.5,0.5}
\definecolor{mauve}{rgb}{0.58,0,0.82}

\lstset{frame=tb,
  language=Java,
  aboveskip=3mm,
  belowskip=3mm,
  showstringspaces=false,
  columns=flexible,
  basicstyle={\small\ttfamily},
  numbers=left,
  numberstyle=\tiny\color{gray},
  keywordstyle=\color{blue},
  commentstyle=\color{dkgreen},
  stringstyle=\color{mauve},
  breaklines=true,
  breakatwhitespace=true,
  tabsize=3
}
\begin{document}

\title{DAT110 Oblig 4 Del A-C}

\author{Joakim Johesan og Eirik Alvestad}

\maketitle

\begin{abstract}
Et arduinobasert cloud IoT adgangskontroll system. Skrevet ved hjelp av TinkerCad, Spark rammeverket og Gson. Fungerer med et REST API mot cloud servicen fra IoT devicet. 
\end{abstract}

\input{./sections/introduction}

\section{Access Control Design Model}
\label{sec:background}

(~1 page) presenting your finite state machine for the access control device that you developed in Part A. This section should contain a figure showing the finite state machine, and contain a short description of it and the main design choices you have made.


Tilstandsmaskinen: \\

Lorenskfodkgldajgiqkdpg



\begin{figure}[H]
  \centering
  \includegraphics[scale=0.75]{figs/StateMachine.png}
  \caption{Adgangskontroll tilstandsmaskin}
  \label{fig:StateMachine}
\end{figure}

\pagebreak

\section{Access Control Hardware/Software Implementation }
(~1.5 pages) explaining how you have implemented the hardware and the software of the access control device. The section should contain a figure presenting your TinkerCAD circuit design

\section{REST API cloud service}
\label{sec:experiments}

(~1.5 page) explaining how you have implemented the cloud service using the Spark/Java framework. It should briefly explain how you have setup the routes in the service, how you have implemented the storage of access codes and the access log, and how you have used Gson.




\section{Device Communication}
\label{sec:prototype}

Device Communication (~1 page) explaining how you have implemented the network communication in the access control device, i.e., the HTTP GET and POST operations on the virtual IoT device.

\section{System Testing}
While we implemented the operations the task required us to implement, it was important for us to be able to test what we just implemented to make sure it was correct. We did this mostly using the postman tool that the task suggested. The postman tool allowed us to easily paste in a the desired url and choose if we wanted to post/get/put to the specified url. The postman tool was the most useful when testing the post method because it allows us to insert a json representation of a object directly in the body of the request.\\
When we were done implementing the cloud service and started implementing the device network communication the testing consisted more of running the device and manually testing the different functions it should be able to perform. But here we also used the postman tool to test the doGetAccessCode() method, because the testing required us to change the access code, and this as easily done using the postman tool.

\section{Conclusions }
(~1/4 page) briefly summing of the status of the project, including things that was not completed or which the group did not manage to get to work properly.

\bibliographystyle{plain}
\bibliography{report.bib}{}

\end{document}